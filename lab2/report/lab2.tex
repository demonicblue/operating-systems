%----------------------------------------------------------------------------------------
%	PACKAGES AND OTHER DOCUMENT CONFIGURATIONS
%----------------------------------------------------------------------------------------

\documentclass[a4paper,11pt,twoside,fleqn]{article}

%\usepackage{fourier} % Use the Adobe Utopia font for the document - comment this line to return to the LaTeX default
\usepackage[english]{babel} % For swedish
%\usepackage{amsmath,amsfonts,amsthm} % Math packages

\usepackage[utf8]{inputenc} % Required for swedish characters
\usepackage[T1]{fontenc}
\usepackage{verbatim}
\usepackage{graphicx}
\graphicspath{{./Graphics/}} % Latex looks for graphics here when including

\usepackage{float}
%\usepackage{fullpage}

\usepackage{booktabs}
\usepackage{tabulary}

%\usepackage{siunitx} % For better units and numbers, best use: \SI{41}{\meter\per\second}

%\usepackage{indentfirst}

\usepackage{gensymb}

\usepackage{sectsty} % Allows customizing section commands
\allsectionsfont{\normalfont} % Make all sections centered, the default font and small caps

\usepackage{amsmath} % Allows for equation indentation
\setlength{\mathindent}{1cm}
\setlength{\parindent}{0pt}

\usepackage{geometry}
\geometry{ % Reference(page 3): ftp://ftp.tex.ac.uk/tex-archive/macros/latex/contrib/geometry/geometry.pdf
  top=3cm,				% Top of page to document body
  inner=3cm,
  outer=3cm,
  bottom=3cm,			% Bottom of page to document body
  headheight=3ex,		% Hard to understand, this value seems sufficient
  headsep=2ex,			% Same
}

%\geometry{showframe=true}	% uncomment this line to see document outlines

% Like cleardoublepage, but reverse logic.
\newcommand*\cleartoleftpage{%
  \clearpage
  \ifodd\value{page}\hbox{}\newpage\fi
}

%%%%%%%%%%%%% NICE HEADERS
\usepackage{fancyhdr} % Fancy header and footer
\fancypagestyle{plain}{
\fancyhead{}
\fancyfoot{}
\fancyhead[RO]{\nouppercase{\rightmark}}
\fancyhead[LE]{\nouppercase{\leftmark}}
\fancyfoot[LE,RO]{\thepage}
\renewcommand{\headrulewidth}{0.4pt}
\renewcommand{\footrulewidth}{0.4pt}
}
\pagestyle{plain}


%\numberwithin{equation}{section} % Number equations within sections (i.e. 1.1, 1.2, 2.1, 2.2 instead of 1, 2, 3, 4)
%\numberwithin{figure}{section} % Number figures within sections (i.e. 1.1, 1.2, 2.1, 2.2 instead of 1, 2, 3, 4)
%\numberwithin{table}{section} % Number tables within sections (i.e. 1.1, 1.2, 2.1, 2.2 instead of 1, 2, 3, 4)

%\setlength\parindent{0pt} % Removes all indentation from paragraphs - comment this line for an assignment with lots of text

%----------------------------------------------------------------------------------------
%	TITLE SECTION
%----------------------------------------------------------------------------------------

%\newcommand{\horrule}[1]{\rule{\linewidth}{#1}} % Create horizontal rule command with 1 argument of height

\title{	
\normalfont \normalsize 
%s\textsc{Chalmers University of Technology} \\ [25pt] % Your university, school and/or department name(s)
\huge EDA092 - Lab 2 Report \\ Group A27 \\ % The assignment title
}

\author{Henrik Hugo \& Simon Fransson \\ (\texttt{hhugo@student.chalmers.se, frsimon@student.chalmers.se})} % Your name

\date{\normalsize\today} % Today's date or a custom date

\begin{document}

\maketitle % Print the title

%----------------------------------------------------------------------------------------
%	PART 1
%----------------------------------------------------------------------------------------

\section{One lock}

%------------------------------------------------

\subsection*{\textbf{Deadlock}}

Question: is deadlock possible to occur? Answer: No.
\\
There are four conditions that needs to be met in order for deadlock to occur:
\begin{itemize}

\item
Mutual exclusion: only one process at a time can use a resource.
\begin{itemize} \item True: Because the \verb+pthread_mutex_lock+ is only allowing one thread at a time to use the queue for enqueuing or dequeuing.
\end{itemize}

\item Hold and wait: a process holding some resource can request additional resources and wait for them if they are held by other processes.
\begin{itemize}
\item False: The enqueue and dequeue functions will not try to request additional resources inside the lock. Also there is only one resource to handle.
\end{itemize}

\item No preemption: a resource can only be release by the process holding it. After that process has completed its task.
\begin{itemize}
\item True: The \verb+pthread_mutex_unlock+ is used for this.
\end{itemize} 

\item Circular wait: there exists a circular chain of 2 or more blocked processes, each waiting for a resource held by the next process in the chain.

\begin{itemize}
\item True: But not in all cases. (4 processes, one resource) 
\end{itemize} 
\end{itemize}

\textbf{Conclusion:} Since at least one of the conditions is broken, deadlock can not occur.
\clearpage

%----------------------------------------------------------------------------------------
%	PART 2
%----------------------------------------------------------------------------------------

\section{Two locks}

%------------------------------------------------

\subsection*{Blah}

Compare against one lock:

\subsection*{\textbf{Deadlock}}

Question: is deadlock possible to occur? Answer: No.
\\
There are four conditions that needs to be met in order for deadlock to occur:
\begin{itemize}

\item
Mutual exclusion: only one process at a time can use a resource.
\begin{itemize} \item True: Because the \verb+pthread_mutex_lock+ is only allowing one thread at a time to use the queue for enqueuing or dequeuing.
\end{itemize}

\item Hold and wait: a process holding some resource can request additional resources and wait for them if they are held by other processes.
\begin{itemize}
\item False: The enqueue and dequeue functions will not try to request additional resources inside the lock. Also there is only one resource to handle.
\end{itemize}

\item No preemption: a resource can only be release by the process holding it. After that process has completed its task.
\begin{itemize}
\item True: The \verb+pthread_mutex_unlock+ is used for this.
\end{itemize} 

\item Circular wait: there exists a circular chain of 2 or more blocked processes, each waiting for a resource held by the next process in the chain.

\begin{itemize}
\item True: But not in all cases. (4 processes, one resource) 
\end{itemize} 
\end{itemize}

\textbf{Conclusion:} Since at least one of the conditions is broken, deadlock can not occur.

%----------------------------------------------------------------------------------------

\end{document}